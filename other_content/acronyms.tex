\begin{acronym}[SEPSEP]
\acro{do}[DO]{dissolved oxygen}
\acro{ena}[ENA]{ecological network analysis}
    \acroplural{ena}[]{ecological network analyses}
\acro{hab}[HAB]{harmful algal bloom}
\acro{so}[SO]{Southern Ocean}
\end{acronym}

% You'll want to list your acronyms in alphabetical order.
% Define an acronym as \acro{<code>}[<acronym>]{<full name>}, 
% e.g. \acro{pom}[POM]{particulate organic matter}
% In the text, use \ac{pom} to refer to the defined acronym, \Ac{pom} to use the acronym but start it with a capital letter, \acf{pom} or \Acf{pom} to force the full name and acronym in brackets, or \acl{pom} or \Acl{pom} to get the full name without the acronym in brackets.
% See https://ctan.org/pkg/acronym?lang=en for more info

% Sometimes the indefinite article of an acronym differs between its short form and its long form, for example `a Federal Bureau of Investigation (FBI) agent' and `an FBI agent'. To deal with this, the package provides the following three commands:
% \acroindefinite{<acronym>}{<short indefinite article>}{<long indefinite article>}
% \newacroindefinite{<acronym>}{<short indefinite article>}{<long indefinite article>}
% \acrodefindefinite{<acronym>}{<short indefinite article>}{<long indefinite article>}
%that allow one to define indefinite articles. The \acroindefinite command is meant to be used in the acronym environment. The difference among the latter two is that \acrodefindefinite puts the acronym definition in the .aux file, so that the acronym exception is available at the next run from start-up. When using \iac and \Iac without first defining an article, the default article is `a'.